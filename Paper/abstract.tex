% From mitthesis package
% Version: 1.01, 2023/06/19
% Documentation: https://ctan.org/pkg/mitthesis
%
% The abstract environment creates all the required headers and footnote. 
% You only need to add the text of the abstract itself.
%
% Approximately 500 words or less; try not to use formulas or special characters
% If you don't want an initial indentation, do \noindent at the start of the abstract

Short-Wave Infrared (SWIR) imaging has become a powerful and well-known technique over the last two decades for silicon inspection and imaging. Recently, developments in defensive and offensive capabilities make such a system an invaluable tool, whether you are a microarchitectural security researcher or chip manufacturer. Traditional methods for silicon inspection or attack, such as "de-capping" (the removal of encapsulating material around a chip die using acid or mechanical means) are often destructive and impractical for consumer verification. Moreover, the imaging techniques used after performing such a destructive evaluation, such as scanning electron microscopy (SEM), X-ray imaging, CT (computed tomography) or optical microscopy are prohibitively expensive and inaccessible to most users. Open-IRIS aims to address these challenges by leveraging a non-destructive, low-cost approach to silicon inspection. Combined with Fourier Ptychographic Microscopy, a technique originally developed for biological imaging applications, high resotion (tens of MP) images can be captured with cheap, hobby-grade optics. Open-IRIS also has applications in research applications, specifically cybersecurity, hardware design, failure analysis. High-precision chip imaging in situ could enable more effective electromagnetic (EM) or laser fault injection attacks, both of which have been successfully used to compromise security on modern cryptographic hardware.